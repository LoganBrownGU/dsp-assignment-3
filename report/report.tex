\documentclass{article}

\usepackage{multicol}
\usepackage{graphicx}
\usepackage{float}
\usepackage{listings}
\usepackage{dirtytalk}
\usepackage{amsmath}
\usepackage{hyperref}
\usepackage{smartref}
\usepackage{svg}
\usepackage[a4paper,margin=2cm]{geometry}

\lstset{basicstyle=\ttfamily\footnotesize,breaklines=true}

\title{DSP Assignment 3 \\ \textit{Wireless Data Transmission Using Lasers}}
\author{Amana Ahmmad, Logan Brown \\ 2590941A, 2641407B}

\begin{document}
\maketitle

\begin{multicols}{2}

\section{Introduction}
Data are usually transmitted wirelessly using sub-infrared frequencies: radio waves, microwaves, and so on. Transmission is usually performed using an omni-directional antenna - this is desirable as the transmitter need not know where the receiver is relative to itself.  

However, there exist use cases where the transmitter may not want to send its data in all directions. If the receiver and transmitter are stationary and in known locations, but it is for some reason not practicable to lay cable between the two, a highly directional wireless system may be more ideal. 

In this case, using a laser to send the data, rather than an antenna, is a possibility. This report explores the use of a laser diode to send data, in conjunction with a phototransistor to detect the signal.

\section{Data framing}
Since the receiver and transmitter are separated, there cannot be a clock line between the two, necessitating the use of an asynchronous scheme. 

To frame the data, UART (universal asynchronous receiver-transmitter) was chosen, as it is commonly used, well documented, and well tested. Python code was written to emulate a UART transmitter (Tx) and receiver (Rx), with two classes that would output 0 or 1, or read 0 or 1 respectively.

The \texttt{UART\_Tx} class allows the user to load an array of bytes, which are then \say{sent} automatically. The \texttt{UART\_Rx} class constantly listens for changes at 8 times the baud rate, and outputs a byte as soon as it determines that it has received one.

\section{Set up}
The laser was aimed at a phototransistor which was set up in a voltage-divider configuration, as outlined in Figure \ref{},\ref{fig:dataflow}. An Arduino connected to a computer measured the voltage across the resistor.

An FSK modulation scheme was used. A logic 0 corresponds to the laser being switched off, effectively 0 Hz, while a logic 1 corresponds to the laser being keyed on and off at a given frequency. This was chosen to be 100 Hz, as it would be significantly higher than any environmental optical and electronic noise, but low enough that it can be reliably sampled at 1000 Hz with minimal aliasing.

\begin{figure}[H]
    \includegraphics[width=\linewidth]{figures/dataflow.pdf}
    \caption{Dataflow diagram}
    \label{fig:dataflow}
\end{figure}

\section{Filtering}
The phototransistor has a relatively wide wavelength response, peaking in near-infrared but extending into shorter-wavelength visible light, and longer infrared. Thus, it responds to much of the ambient light in an environment, and so there is a significant amount of noise in the signal -- both optical and electrical.

The main sources of noise are the 50 Hz mains electric hum, very low-frequency drift from natural sources, and DC drift from the ambient light level. 

To remove these unwanted frequencies, an 8\textsuperscript{th} order (or rather, a chain of 4 2\textsuperscript{nd} order) bandpass IIR filter was used, with the passband being 5 Hz either side of the sending frequency. 

\section{Baud rate}
Finally, the baud rate of the UARTs was set to 10. The delay of the IIR filter and the response of the phototransistor were slow enough that sending faster than this would not allow the logic levels enough time to settle, resulting in missed bits. Naturally, this would be too slow for a real world scenario, but with more research and engineering it would be relatively trivial to make this faster. 

\section{Conclusion}
It was possible to send large amounts of data with no loss, albeit quite slowly. The IIR filter performed well, eliminating all environmental noise and providing a clean signal to the UART receiver, though there was a significant delay in the response that limited the baud rate.



\end{multicols}
\pagebreak

\section{Appendix}

\subsection{uart.py}
\lstinputlisting[language=python]{../python/uart.py}

\subsection{filter.py}
\lstinputlisting[language=python]{../python/filter.py}

\subsection{graph.py}
\lstinputlisting[language=python]{../python/graph.py}

\subsection{receiver.py}
\lstinputlisting[language=python]{../python/receiver.py}

\subsection{transmitter.py}
\lstinputlisting[language=python]{../python/transmitter.py}

\subsection{config.py}
\lstinputlisting[language=python]{../python/config.py}


\end{document}